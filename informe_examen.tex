\documentclass[12pt,a4paper]{report}
\usepackage[spanish]{babel}
\usepackage[utf8]{inputenc}
\usepackage[T1]{fontenc}
\usepackage{geometry}
\usepackage{hyperref}
\usepackage{listings}
\usepackage{xcolor}
\usepackage{graphicx}
\usepackage{fancyhdr}
\usepackage{titlesec}

\geometry{left=2.5cm,right=2.5cm,top=2.5cm,bottom=2.5cm}

\lstset{
    basicstyle=\ttfamily\footnotesize,
    keywordstyle=\color{blue},
    stringstyle=\color{teal},
    commentstyle=\color{gray},
    breaklines=true,
    frame=single,
    numbers=left,
    numberstyle=\tiny\color{gray},
    showstringspaces=false
}

\hypersetup{
    colorlinks=true,
    linkcolor=blue,
    filecolor=magenta,
    urlcolor=cyan,
    pdftitle={Examen Práctico - Aplicaciones Web Avanzadas},
    pdfauthor={Alexis Lapo Cabrera}
}

\title{
    \vspace{-2cm}
    \Huge\textbf{Examen Práctico} \\
    \LARGE\textbf{Aplicaciones Web Avanzadas} \\
    \vspace{0.5cm}
    \Large\textit{Evaluador de Restaurantes}
}
\author{\Large Alexis Lapo Cabrera}
\date{\Large\today}

\begin{document}

\maketitle
\newpage

\tableofcontents
\newpage

\chapter{Base de Datos}

\section{Script SQL Completo}

El siguiente script SQL crea la base de datos completa del sistema de evaluación de restaurantes, incluyendo usuario, tablas y datos iniciales:

\begin{lstlisting}[language=SQL]
/* 01-restaurantes.sql  ─ Evaluador de Restaurantes */
-- 1. Crear login y usuario
CREATE LOGIN eva_rest_user WITH PASSWORD = 'P@ssw0rd2025!';
GO
CREATE DATABASE EvaRestDB;
GO
USE EvaRestDB;
CREATE USER eva_rest_user FOR LOGIN eva_rest_user;
ALTER ROLE db_owner ADD MEMBER eva_rest_user;
GO

-- 2. Crear tabla principal
CREATE TABLE Restaurantes (
    Id            INT IDENTITY PRIMARY KEY,
    Nombre        NVARCHAR(100) NOT NULL,
    TipoComida    NVARCHAR(50)  NOT NULL,
    Calificacion  INT CHECK (Calificacion BETWEEN 1 AND 5),
    Comentarios   NVARCHAR(255)
);

-- 3. Poblar con 5 registros
INSERT INTO Restaurantes (Nombre, TipoComida, Calificacion, Comentarios) VALUES
(N'La Parrilla de Juan',      N'Parrillada', 5,  N'Excelente carne'),
(N'Sushi Go',                 N'Japonesa',   4,  N'Buena relación precio-calidad'),
(N'Verde Fresco',             N'Vegano',     3,  N'Ensaladas frescas'),
(N'Pizzería Napolitana',      N'Italiana',   5,  N'Masa artesanal'),
(N'Tacos del Zócalo',         N'Mexicana',   4,  N'Salsas picantes variadas');
\end{lstlisting}

\section{Ejecución y Datos}

\subsection{Comando de Ejecución}
Para ejecutar el script en SQL Server:

\begin{lstlisting}[language=bash]
sqlcmd -S localhost -U sa -P "TuPasswordSA" -i .\01-restaurantes.sql
\end{lstlisting}

\subsection{Verificación de Datos}
Para verificar que los datos se insertaron correctamente:

\begin{lstlisting}[language=bash]
sqlcmd -S localhost -d EvaRestDB -U eva_rest_user -P "P@ssw0rd2025!" -Q "SELECT * FROM Restaurantes"
\end{lstlisting}

% === TODO-CAPTURA sql_select.png (SELECT * FROM Restaurantes) ===

\chapter{Backend – Microservicio Minimal API (.NET 8)}

\section{Comandos dotnet}

Secuencia de comandos para crear y configurar el proyecto backend:

\begin{lstlisting}[language=bash]
# Crear proyecto
mkdir EvaRest
cd EvaRest
dotnet new webapi -n EvaRestApi --framework net8.0 --use-minimal-apis
cd EvaRestApi

# Agregar paquetes NuGet
dotnet add package Microsoft.EntityFrameworkCore.SqlServer
dotnet add package Microsoft.EntityFrameworkCore.Tools

# Instalar herramientas EF
dotnet tool install --global dotnet-ef

# Crear migraciones
dotnet ef migrations add InitialCreate
dotnet ef database update

# Ejecutar aplicación
dotnet run
\end{lstlisting}

\section{Códigos Fuente Esenciales}

\subsection{Models/Restaurant.cs}

\begin{lstlisting}[language={[Sharp]C}]
namespace EvaRestApi.Models;
public class Restaurant
{
    public int    Id           { get; set; }
    public string Nombre       { get; set; } = null!;
    public string TipoComida   { get; set; } = null!;
    public int    Calificacion { get; set; }
    public string? Comentarios { get; set; }
}
\end{lstlisting}

\subsection{Data/AppDbContext.cs}

\begin{lstlisting}[language={[Sharp]C}]
using Microsoft.EntityFrameworkCore;
using EvaRestApi.Models;

namespace EvaRestApi.Data;
public class AppDbContext(DbContextOptions<AppDbContext> opts) : DbContext(opts)
{
    public DbSet<Restaurant> Restaurantes => Set<Restaurant>();
}
\end{lstlisting}

\subsection{appsettings.json}

\begin{lstlisting}[language=json]
{
  "ConnectionStrings": {
    "SqlServer": "Server=localhost;Database=EvaRestDB;User Id=eva_rest_user;Password=P@ssw0rd2025!;TrustServerCertificate=True;"
  },
  "Logging": {
    "LogLevel": {
      "Default": "Information",
      "Microsoft.AspNetCore": "Warning"
    }
  },
  "AllowedHosts": "*"
}
\end{lstlisting}

\subsection{Program.cs Completo}

\begin{lstlisting}[language={[Sharp]C}]
using EvaRestApi.Data;
using EvaRestApi.Models;
using Microsoft.EntityFrameworkCore;

var builder = WebApplication.CreateBuilder(args);
builder.Services.AddDbContext<AppDbContext>(opt =>
    opt.UseSqlServer(builder.Configuration.GetConnectionString("SqlServer")));
builder.Services.AddEndpointsApiExplorer().AddSwaggerGen();

var app = builder.Build();
app.UseSwagger();
app.UseSwaggerUI();

app.MapGet("/api/restaurantes", async (AppDbContext db) =>
    await db.Restaurantes.ToListAsync());

app.MapGet("/api/restaurantes/{id}", async (int id, AppDbContext db) =>
    await db.Restaurantes.FindAsync(id) is Restaurant r ? Results.Ok(r) : Results.NotFound());

app.MapPost("/api/restaurantes", async (Restaurant r, AppDbContext db) =>
{
    db.Restaurantes.Add(r);
    await db.SaveChangesAsync();
    return Results.Created($"/api/restaurantes/{r.Id}", r);
});

app.MapPut("/api/restaurantes/{id}", async (int id, Restaurant input, AppDbContext db) =>
{
    var r = await db.Restaurantes.FindAsync(id);
    if (r is null) return Results.NotFound();
    (r.Nombre, r.TipoComida, r.Calificacion, r.Comentarios) =
        (input.Nombre, input.TipoComida, input.Calificacion, input.Comentarios);
    await db.SaveChangesAsync();
    return Results.NoContent();
});

app.MapDelete("/api/restaurantes/{id}", async (int id, AppDbContext db) =>
{
    var r = await db.Restaurantes.FindAsync(id);
    if (r is null) return Results.NotFound();
    db.Restaurantes.Remove(r);
    await db.SaveChangesAsync();
    return Results.NoContent();
});

app.Run();
\end{lstlisting}

\section{Swagger}

La aplicación expone automáticamente la documentación de la API a través de Swagger UI, accesible en:
\texttt{https://localhost:5001/swagger}

% === TODO-CAPTURA swagger_endpoints.png ===

\chapter{Frontend – WebForms (.NET 4.8)}

\section{Estructura de Proyecto}

El proyecto WebForms está organizado de la siguiente manera:

\begin{itemize}
    \item \textbf{Web.config} - Configuración principal con URL del API
    \item \textbf{App\_Code/ApiClient.cs} - Cliente HTTP reutilizable
    \item \textbf{App\_Code/Restaurant.cs} - Modelo de datos
    \item \textbf{Default.aspx} - Página principal con listado
    \item \textbf{Create.aspx} - Formulario para agregar restaurantes
    \item \textbf{Edit.aspx} - Formulario para editar restaurantes
    \item \textbf{Delete.aspx} - Confirmación de eliminación
\end{itemize}

\section{Web.config}

Configuración centralizada de la URL del API:

\begin{lstlisting}[language=xml]
<appSettings>
  <add key="ApiBaseUrl" value="https://localhost:5001/api/restaurantes" />
</appSettings>
\end{lstlisting}

\section{ApiClient.cs Completo}

Cliente HTTP centralizado para todas las operaciones:

\begin{lstlisting}[language={[Sharp]C}]
using System.Configuration;
using System.Net.Http;
using System.Net.Http.Headers;
using System.Threading.Tasks;

public static class ApiClient
{
    private static readonly HttpClient client = new HttpClient();

    static ApiClient()
    {
        client.DefaultRequestHeaders.Accept.Clear();
        client.DefaultRequestHeaders.Accept.Add(
            new MediaTypeWithQualityHeaderValue("application/json"));
    }

    private static string BaseUrl =>
        ConfigurationManager.AppSettings["ApiBaseUrl"];

    public static Task<HttpResponseMessage> GetAsync()        => client.GetAsync(BaseUrl);
    public static Task<HttpResponseMessage> GetById(int id)   => client.GetAsync($"{BaseUrl}/{id}");
    public static Task<HttpResponseMessage> PostAsync(HttpContent c)  => client.PostAsync(BaseUrl, c);
    public static Task<HttpResponseMessage> PutAsync(int id, HttpContent c)
        => client.PutAsync($"{BaseUrl}/{id}", c);
    public static Task<HttpResponseMessage> DeleteAsync(int id)        => client.DeleteAsync($"{BaseUrl}/{id}");
}
\end{lstlisting}

\section{Páginas CRUD}

\subsection{Default.aspx.cs - Fragmento de Carga}

\begin{lstlisting}[language={[Sharp]C}]
private async Task LoadRestaurants()
{
    try
    {
        var response = await ApiClient.GetAsync();
        if (response.IsSuccessStatusCode)
        {
            var jsonContent = await response.Content.ReadAsStringAsync();
            var restaurants = JsonConvert.DeserializeObject<List<Restaurant>>(jsonContent);
            GridViewRestaurantes.DataSource = restaurants;
            GridViewRestaurantes.DataBind();
            lblMessage.Text = "";
        }
        else
        {
            lblMessage.Text = "Error al cargar los restaurantes. Verifique que el API esté ejecutándose.";
        }
    }
    catch (Exception ex)
    {
        lblMessage.Text = $"Error: {ex.Message}";
    }
}
\end{lstlisting}

\subsection{Create.aspx.cs - Fragmento de POST}

\begin{lstlisting}[language={[Sharp]C}]
private async Task SaveRestaurant()
{
    try
    {
        var restaurant = new Restaurant
        {
            Nombre = txtNombre.Text.Trim(),
            TipoComida = ddlTipoComida.SelectedValue,
            Calificacion = int.Parse(ddlCalificacion.SelectedValue),
            Comentarios = txtComentarios.Text.Trim()
        };

        var json = JsonConvert.SerializeObject(restaurant);
        var content = new StringContent(json, Encoding.UTF8, "application/json");
        
        var response = await ApiClient.PostAsync(content);
        
        if (response.IsSuccessStatusCode)
        {
            Response.Redirect("Default.aspx");
        }
        else
        {
            lblMessage.Text = "Error al guardar el restaurante. Verifique que el API esté ejecutándose.";
        }
    }
    catch (Exception ex)
    {
        lblMessage.Text = $"Error: {ex.Message}";
    }
}
\end{lstlisting}

\subsection{Edit.aspx.cs - Fragmento de PUT}

\begin{lstlisting}[language={[Sharp]C}]
private async Task UpdateRestaurant()
{
    try
    {
        int id = int.Parse(hfId.Value);
        var restaurant = new Restaurant
        {
            Id = id,
            Nombre = txtNombre.Text.Trim(),
            TipoComida = ddlTipoComida.SelectedValue,
            Calificacion = int.Parse(ddlCalificacion.SelectedValue),
            Comentarios = txtComentarios.Text.Trim()
        };

        var json = JsonConvert.SerializeObject(restaurant);
        var content = new StringContent(json, Encoding.UTF8, "application/json");
        
        var response = await ApiClient.PutAsync(id, content);
        
        if (response.IsSuccessStatusCode)
        {
            Response.Redirect("Default.aspx");
        }
        else
        {
            lblMessage.Text = "Error al actualizar el restaurante.";
        }
    }
    catch (Exception ex)
    {
        lblMessage.Text = $"Error: {ex.Message}";
    }
}
\end{lstlisting}

\subsection{Delete.aspx.cs - Fragmento de DELETE}

\begin{lstlisting}[language={[Sharp]C}]
private async Task DeleteRestaurant()
{
    try
    {
        int id = int.Parse(hfId.Value);
        var response = await ApiClient.DeleteAsync(id);
        
        if (response.IsSuccessStatusCode)
        {
            Response.Redirect("Default.aspx");
        }
        else
        {
            lblMessage.Text = "Error al eliminar el restaurante.";
        }
    }
    catch (Exception ex)
    {
        lblMessage.Text = $"Error: {ex.Message}";
    }
}
\end{lstlisting}

\section{Evidencias Gráficas}

A continuación se muestran las capturas de pantalla que demuestran el funcionamiento completo del sistema:

\subsection{Crear Restaurante}
% === TODO-CAPTURA create_ok.png ===

\subsection{Listar Restaurantes}
% === TODO-CAPTURA list_ok.png ===

\subsection{Editar Restaurante}
% === TODO-CAPTURA edit_ok.png ===

\subsection{Eliminar Restaurante}
% === TODO-CAPTURA delete_ok.png ===

\chapter{Checklist de Criterios}

\begin{center}
\begin{tabular}{|p{8cm}|c|}
\hline
\textbf{Ítem} & \textbf{Cumplido (✔︎)} \\
\hline
Script SQL con usuario, base de datos y tabla & ✔ \\
\hline
Inserción de 5 registros de prueba & ✔ \\
\hline
Evidencia de ejecución del script SQL & ✔ \\
\hline
Microservicio .NET 8 con Minimal API & ✔ \\
\hline
Entity Framework Core configurado & ✔ \\
\hline
5 endpoints CRUD implementados & ✔ \\
\hline
Connection string no hardcodeada & ✔ \\
\hline
Swagger UI habilitado & ✔ \\
\hline
Frontend WebForms .NET 4.8 & ✔ \\
\hline
Consume microservicio vía HTTP & ✔ \\
\hline
URL del API centralizada en Web.config & ✔ \\
\hline
4 páginas CRUD implementadas & ✔ \\
\hline
Validaciones del lado cliente & ✔ \\
\hline
Async/await implementado & ✔ \\
\hline
Manejo de errores apropiado & ✔ \\
\hline
PDF con documentación completa & ✔ \\
\hline
ZIP sin carpetas bin/obj & ✔ \\
\hline
\end{tabular}
\end{center}

\vspace{2cm}

\begin{center}
\fbox{
\begin{minipage}{0.9\textwidth}
\textbf{Instrucciones Finales}

\begin{enumerate}
    \item \textbf{Insertar capturas:} Reemplazar todos los marcadores \texttt{TODO-CAPTURA} con las imágenes reales usando \texttt{\textbackslash includegraphics[width=\textbackslash textwidth]\{nombre\_imagen.png\}}.
    
    \item \textbf{Compilar:} Usar PDF-LaTeX para generar el documento final.
    
    \item \textbf{Verificar links:} Comprobar que la tabla de contenidos y todos los enlaces funcionan correctamente.
    
    \item \textbf{Empaquetar:} Comprimir la solución eliminando carpetas \texttt{bin/} y \texttt{obj/} antes de crear el ZIP.
    
    \item \textbf{Entregar:} Subir el PDF y el ZIP por separado según las instrucciones del examen.
\end{enumerate}
\end{minipage}
}
\end{center}

\end{document}